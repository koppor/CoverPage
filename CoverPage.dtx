% \iffalse
%
% CoverPage.dtx
% Docstrip archive, run through LaTeX.
%
% Copyright (C) 2006 by Matthias M�hlich
%                       (matthias.muehlich@lfb.rwth-aachen.de)
% 
% This file may be distributed and/or modified under the
% conditions of the LaTeX Project Public License, either version 1.2
% of this license or (at your option) any later version.
% The latest version of this license is in:
%
%    http://www.latex-project.org/lppl.txt
%
% and version 1.2 or later is part of all distributions of LaTeX 
% version 1999/12/01 or later.
%
%<*driver>
\ProvidesFile{CoverPage.dtx}%
  [2006/02/23 v1.01  automatic cover page generation  (MM)]
\documentclass[a4paper]{ltxdoc}
\usepackage{textcomp}
\RecordChanges
%%\OnlyDescription
\begin{document}
  \DocInput{CoverPage.dtx}
\end{document}
%</driver>
%
% \fi
%
%
% \CheckSum{0}
%
% \CharacterTable
%  {Upper-case    \A\B\C\D\E\F\G\H\I\J\K\L\M\N\O\P\Q\R\S\T\U\V\W\X\Y\Z
%   Lower-case    \a\b\c\d\e\f\g\h\i\j\k\l\m\n\o\p\q\r\s\t\u\v\w\x\y\z
%   Digits        \0\1\2\3\4\5\6\7\8\9
%   Exclamation   \!     Double quote  \"     Hash (number) \#
%   Dollar        \$     Percent       \%     Ampersand     \&
%   Acute accent  \'     Left paren    \(     Right paren   \)
%   Asterisk      \*     Plus          \+     Comma         \,
%   Minus         \-     Point         \.     Solidus       \/
%   Colon         \:     Semicolon     \;     Less than     \<
%   Equals        \=     Greater than  \>     Question mark \?
%   Commercial at \@     Left bracket  \[     Backslash     \\
%   Right bracket \]     Circumflex    \^     Underscore    \_
%   Grave accent  \`     Left brace    \{     Vertical bar  \|
%   Right brace   \}     Tilde         \~}
%
%
%    \changes{v1.00}{2006/02/20}{Initial version}
%    \changes{v1.01}{2006/02/23}{Names of package and macros changed as
%      requested by CTAN team. Support for publisher SPIE added.}
%    \GetFileInfo{CoverPage.dtx}
%
%
%    \title{The \texttt{CoverPage} Package\thanks{This document
%       was generated from \texttt{CoverPage.dtx}~\fileversion, dated \filedate.}}
%    \author{Matthias M\"uhlich \\ \texttt{matthias.muehlich@lfb.rwth-aachen.de}}
%    \date{version 1.01 \quad 23-FEB-2006}
%    \maketitle
%
%    \frenchspacing
%    \sloppy
%
%    \begin{abstract}
%       The package \texttt{CoverPage} was created to supplement
%       scientific papers with a cover page containing bibliographical
%       information, a copyright notice, and/or some logos of the author's
%       institution. The cover page is created (almost) automatically; this 
%       is done by parsing \BibTeX{} information corresponding to the
%       main document and reading a configuration file in which the author
%       can set information like the affiliation he or she is associated with.
%       
%       The cover page consists of header, body and footer; all three are 
%       macros which can be redefined using |\renewcommand|, thus allowing
%       easy customization of the package. Additionally, it should be stressed 
%       that the cover page layout is totally \emph{independent} of the 
%       main document and its page layout 
%       (which is not the easiest thing to achieve in \LaTeX).
%
%       This package requires four other packages (\texttt{keyval}, \texttt{url},
%       \texttt{textcomp}, and \texttt{verbatim}), but all of them are standard 
%       packages and should be part of every \LaTeX{} installation.
%    \end{abstract}
%    
%    
%    \tableofcontents
%    
%    
%    \section*{Foreword}
%
%    The best explanation for this package is an example. So first of all, 
%    please look at the file \texttt{ECCV2006Sample.pdf} which is part of 
%    this distribution. The cover page was created using 
%    the \texttt{CoverPage} package, and adding the cover page in front of the 
%    document required only adding a single line of \LaTeX{} code to the original 
%    source file.
%   
%   
%   
%   
%   
%    \section{About this Package}
%   
%    Researchers (like the author of this package) want to distribute their own work in the 
%    scientific community, either as printout or in electronic form (email or 
%    download). The scientific paper itself is sufficient for this purpose, but 
%    it is not optimal for several reasons:
%    \begin{itemize}
%    \item
%       Other scientists need \emph{bibliographical information} on the paper 
%       in order to cite it properly. So let us be nice to our colleagues and 
%       attach this necessary information to the paper in some way.
%    \item
%       One should not ignore the legal subtleties. If papers are published in 
%       some journal or conference proceedings, one does not own the copyright 
%       anymore; this is especially problematic if the paper shall be 
%       available for download. Fortunately, most publishers have now realized 
%       that researchers \emph{need} to put their work on their homepages 
%       (otherwise, the work is widely ignored). Therefore, most publishers 
%       allow redistribution for scientific use provided that some special 
%       \emph{copyright notice} is attached to the paper.
%    \item
%       Last, but not least, we want our colleagues to recognize the logos of our
%       excellent work group and our great university. Scientific papers are 
%       characterized by highly different page layouts, while a unified look of 
%       cover pages together with logos can help to establish a 
%       recognizable \emph{corporate design}.
%    \end{itemize}
%    All three aims can be reached by supplementing the paper with an additional cover 
%    page which contains all the required information. This is exactly the purpose 
%    of the \texttt{CoverPage} package -- and it does this job in a widely automated way.
%
%
%
%
%
%    \section{Installation}
%
%    The package \texttt{CoverPage} is distributed as \texttt{dtx} archive together
%    with a corresponding \texttt{ins} installation script. Advanced \LaTeX{} users 
%    who know these file types may now install the package in the usual \LaTeX{} way
%    and skip the remainder of this section.
%    
%    For all others: \texttt{dtx} files are text files which combine \LaTeX{} code with 
%    its own code documentation. The whole \LaTeX{} base system and almost all standard
%    packages are distributed in this form. In order to install such files, you must
%    \begin{enumerate}
%    \item
%      Enter \texttt{latex CoverPage.ins} at the command line. This extracts all 
%      necessary files from the archive (here: the \texttt{CoverPage.sty} style file
%      and a \texttt{CoverPage.cfg} configuration file.
%    \item{}
%      [optionally:] Enter \texttt{latex CoverPage.dtx} at the command line. 
%      This produces a documentation of the package (user interface and code documentation). 
%      ^^A Package documentations often include their own \texttt{sty} file for demonstration
%      ^^A purposes, so the order usually \emph{is} important. However, this is not the case for 
%      ^^A \texttt{CoverPage}.
%    \end{enumerate}
%    Note that the documentation \texttt{CoverPage.pdf} which is distributed with the 
%    package only contains the user interface. If you also want to take a look at the 
%    commented source code, you have to execute the step described above.
%    
%    
%    \section{Setup and Usage}
%
%    This package can do its job more or less automatically, but you also have access 
%    to every setting if you want to modify some behaviour by hand.
%    
%    
%    \subsection{Automatic cover page creation}
%
%    Ideally, you have to include just one line in your \LaTeX{} document:
%    \begin{verbatim}
%    \usepackage{CoverPage}
%    \end{verbatim}
%    That's all. But in order to make automatic cover page creation
%    work properly, you have to do two things in advance:
%    \begin{itemize}
%    \item
%       You have to create a text file containg the \BibTeX{} entry of your document. 
%       The file name must be \meta{mainTeXFileName}\texttt{.BibTeX.txt}. For instance, if
%       you entered \texttt{pdflatex myPaper.tex} on the command line (or let some
%       specialized editor issue this command), then 
%       the corresponding bibliographical information (and nothing more!) should be put in 
%       the file \texttt{myPaper.BibTeX.txt}.
%    \item
%       Even after parsing the \BibTeX{} data, the \texttt{CoverPage} package cannot know 
%       your institute, postal adress, or university logo. Fortunately, this missing data 
%       is not specific to the document, but can be specified once and for all, 
%       namely in the \texttt{CoverPage.cfg} configuration file.
%    \end{itemize}
%    As the cover page is directly printed at the |\begin{document}| command, 
%    it is clear that all cover page data settings must appear \emph{before} this command, i.e. 
%    in the preamble. There are three different ways how texts get on the cover page: 
%    parsing of the \BibTeX{} entry, settings in the config file \texttt{CoverPage.cfg}, and user input in 
%    the document preamble (in this order). Note that later settings can 
%    overwrite previous ones.
%
%
%    \subsection{Options}
%
%    By using |\usepackage[|\meta{optionList}|]{CoverPage}|, the user can specify 
%    package options. Four such options are currently supported.
%
%    The language options \texttt{german} and \texttt{english} set some language-specific string in 
%    auto-generated document source strings. If no language is specified, it defaults to English.
%    
%    The option \texttt{twoside} is intended for two-sided printouts. In this case, we want the
%    original document to start on an odd (right) page, so the package will print the cover page
%    followed by a further empty page (the back side of the cover page).
%    
%    The last possible option is \texttt{nobib}. If no \BibTeX{} file exists, this package will 
%    stop with an error. The |nobib| option disables this behaviour and allows to proceed
%    normally. Then, the cover will not contain a \BibTeX{} reference and the user will have to 
%    set all texts manually because no \BibTeX{} file can be parsed for the relevant data.
%
%
%    \subsection{Setting cover page data by hand}
%
%    Setting of cover page data either in the config file or in the document preamble
%    is done with the same commands. In fact, it is just one single command (and a 
%    few macros, like fonts, which can be redefined using |\renewcommand|).
%
%    \DescribeMacro{\CoverPageSetup}
%    The command |\CoverPageSetup{|\meta{CommaSeparatedList}|}| uses the 
%    \texttt{keyval} package to provide a user-friendly interface for setup of 
%    the cover page. The \meta{CommaSeparatedList} consist of pairs 
%    \meta{key}|=|\meta{value}. If \meta{value} contains one or more 
%    commas, it must be enclosed in braces. A good example is the setting of the 
%    institute which can be centralized in the config file:
%    \begin{verbatim}
%    \CoverPageSetup{institute =
%    {Institute of Imaging and Computer Vision\\
%     RWTH Aachen University, 52056 Aachen, Germany\\
%     tel: +49 241 80 27860, fax: +49 241 80 22200\\
%     web: www.lfb.rwth-aachen.de}}
%    \end{verbatim}
%    Allowed keys are:
%    \begin{itemize}
%    \item
%       The keys |title|, |author|, |institute|, |insource|, and 
%       |copyright| set the corresponding texts in the cover page.
%    \item
%       An automatic copyright notice can be created with |year = |\meta{year} 
%       and a known |publisher| key. In the current version, the package 
%       recognizes |Springer| (copyright text is correct for lecture notes
%       on computer science (LNCS) series), |Kluwer|, |SPIE| and |IEEE|. For these 
%       publishers, the copyright notice requested by the corresponding publisher is 
%       created automatically.
%    \item
%       An automatic source information is created if the keys |booktitle| or 
%       |journal| are defined. In this case, the source text reads: ``in: 
%       \meta{booktitle$\vert$journal}. \meta{referenceToBibTeXEntry}'' where 
%       \meta{referenceToBibTeXEntry} is a language dependent text. The 
%       current version supports the package options \texttt{german} and 
%       \texttt{english} (default).
%    \end{itemize}
%    Note that parsing the \BibTeX{} entry routes the information through the same 
%    \texttt{keyval} interface which is also used for the manual setup. 
%    The above information on the automatic generation 
%    of copyright or source information also holds if the \BibTeX{} entry contains 
%    the necessary keywords. Currently supported \BibTeX{} type identifiers are 
%    |@article|, |@inproceedings|, |@incollection|, and |@misc|.
%
%
%    \subsection{Customization}
%
%    \DescribeMacro{\CPTitleFont}
%    \DescribeMacro{\CPAuthorFont}
%    \DescribeMacro{\CPInstituteFont}
%    \DescribeMacro{\CPInSourceFont}
%    \DescribeMacro{\CPCopyrightFont}
%    The five texts for title, author, institute, source, and copyright are all 
%    associated with an individual font. The standard definitions are
%    \begin{verbatim}
%    \newcommand{\CPTitleFont}{\Huge\sffamily\bfseries\upshape}
%    \newcommand{\CPAuthorFont}{\LARGE\sffamily\mdseries\upshape}
%    \newcommand{\CPInstituteFont}{\Large\sffamily\mdseries\upshape}
%    \newcommand{\CPInSourceFont}{\large\sffamily\mdseries\upshape}
%    \newcommand{\CPCopyrightFont}{\large\sffamily\mdseries\upshape}
%    \end{verbatim}
%    but you can change these setting with |\renewcommand|.
%
%    \DescribeMacro{\CoverPageHeader}
%    \DescribeMacro{\CoverPageBody}
%    \DescribeMacro{\CoverPageFooter}
%    \DescribeMacro{\CoverPageFooterLogo}
%    The cover page consists of three parts: header, body, footer. These parts are 
%    encapsulated as macros which allow redefinition. For instance, the header is 
%    empty by default, but
%    \begin{verbatim}
%    \renewcommand{\CoverPageHeader}{%
%       \includegraphics[width=\textwidth]{lfbheader}}
%    \end{verbatim}
%    tells \texttt{CoverPage} to use a graphics, scaled to full width. The same holds for 
%    |\CoverPageBody| and |\CoverPageFooter| (cf. code documentation at the 
%    end for the default definitions). The footer, however, contains more or less 
%    fixed parts (\BibTeX{} info and copyright notice), so it offers another point of
%    entry for easy user customization: The macro |\CoverPageFooterLogo| defines the 
%    contents of the lower right corner; for instance, this could be used to display 
%    a university logo.
%
%    Costumization commands which are needed on a regular basis should be put in 
%    the config file \texttt{CoverPage.cfg}. Please do not modify the style file
%    \texttt{CoverPage.sty} directly.
%
%
%
%
% \StopEventually{}
%
%
%
%
%    \section{Implementation Part I -- Initialization}
%
%    From now on, we describe the code of the file \texttt{CoverPage.dtx}.
%    It is mainly intended for the advanced \LaTeX{} user who wants to understand
%    how this package works.
%
%
%    \subsection{Identification}
%
%    The first line indicates the begin of the \texttt{CoverPage.sty} code.
%    Then we check that the \LaTeX{} version is not too old and the package
%    identifies itself.
%
%    \begin{macrocode}
%<*CoverPage>
\NeedsTeXFormat{LaTeX2e}[1999/12/01]
\ProvidesPackage{CoverPage}%
  [2006/02/23 v1.01  automatic cover page generation  (MM)]
%    \end{macrocode}
%
%
%    \subsection{Required Packages}
%
%    We need the following packages:
%    \begin{itemize}
%    \item
%      \texttt{keyval}: our package can be customized using the \texttt{keyval}
%      interface, which is e.g. known from the \texttt{graphicx} package.
%    \item
%      \texttt{textcomp}: mainly for |\textcopyright| (the \textcopyright{} symbol)
%    \item
%      \texttt{url}: for typesetting URLs.
%    \item
%      \texttt{verbatim}: listing of the \BibTeX{} entry.
%    \end{itemize}
%    These are the next four lines of our code:
%    \begin{macrocode}
\RequirePackage{keyval}
\RequirePackage{textcomp}
\RequirePackage{url}
\RequirePackage{verbatim}
%    \end{macrocode}
%
%
%    \subsection{Switches and options}
%
%    Now we can implement the options.
%    The macros |\CP@BibRef| (default: ``See also \BibTeX{} entry below.'') 
%    and |\CP@In| (default: ``in:'') are initialized in English and can be
%    overwritten by language options (no further action is necessary for |\CP@In| 
%    as German and English, the only supported languages in this version, are
%    identical here).
%
%    For the option |twoside| (two-sided printout; cover page must get a blank
%    back side), we define a command which is to be appended at the end of 
%    the cover page creation macro. By default, it does nothing, but with the 
%    |twoside| option, we pass an empty box to |\shipout| instead, thus printing
%    an empty page. Note that global options are automatically passed on to 
%    packages, so specifying |twoside| as option in |\documentclass| should be
%    sufficient.
%    \begin{macrocode}
\newif\ifCP@nobib\CP@nobibfalse
\gdef\CP@BibRef{See also \BibTeX{} entry below.}
\gdef\CP@In{in:}
\let\CP@AdvancePageTwoSided\relax
\DeclareOption{german}{\gdef\CP@BibRef{Siehe auch \BibTeX{} Eintrag unten.}}
\DeclareOption{english}{\gdef\CP@BibRef{See also \BibTeX{} entry below.}}
\DeclareOption{twoside}{\gdef\CP@AdvancePageTwoSided{\setbox255\vbox{}\shipout\box255}}
\DeclareOption{nobib}{\CP@nobibtrue}
\ProcessOptions\relax
%    \end{macrocode}
%
%
%    \subsection{New register definitions}
%
%    Our cover page shall have 0.75 inch margin on each side. Width and height are defined
%    in the two dimension registers |\CP@CoverPageWidth| and |\CP@CoverPageHeight| 
%    which we first define and then compute from the total paper size.
%    
%    \begin{macrocode}
\newdimen\CP@CoverPageWidth
\newdimen\CP@CoverPageHeight
\CP@CoverPageWidth=\paperwidth
\CP@CoverPageHeight=\paperheight
\advance\CP@CoverPageWidth by -1.5in
\advance\CP@CoverPageHeight by -1.5in
%    \end{macrocode}
%
%
%    \subsection{Text macros}
%
%    The text on the cover page consists of five macros: title, author, institute, the 
%    source where the paper can be found and a copyright notice.
%
%    \begin{macrocode}
\def\CP@Title{title undefined}
\def\CP@Author{author undefined}
\def\CP@Institute{institute undefined}
\def\CP@InSource{source undefined}
\def\CP@Copyright{\textcopyright{} copyright by the author(s)}
%    \end{macrocode}
%
%    Additional, we define macros for constructing source and copyright notice
%    automatically.
%
%    \begin{macrocode}
\def\CP@Year{(unknown year)}
\def\CP@Source{(unknown source)}
%    \end{macrocode}
%
%    At the end, we `steal' a definition from the \texttt{doc} package.
%
%    \begin{macrocode}
\@ifundefined{BibTeX}
   {\def\BibTeX{{\rmfamily B\kern-.05em%
    \textsc{i\kern-.025em b}\kern-.08em%
    T\kern-.1667em\lower.7ex\hbox{E}\kern-.125emX}}}{}
%    \end{macrocode}
%
%
%
%
%
%    \section{Implementation Part II -- Internal Commands}
%
%    This section contains some internal commands defined in \texttt{CoverPage}.
%    These macros all start with |CP@|; please do not use them because they
%    are not part of the user interface and might be redefined in future versions.
%
%
%    \subsection{Define and read bibliography file}
%    
%    If \texttt{nobib} option is not set, do the following:
%    Check if file exists. If so, store argument of |@inproceedings| or 
%    |@article| for later parsing. In order to achieve this, we temporarily (i.e. 
%    locally) define macros |\inproceedings| and |\article| and tell \LaTeX{} 
%    that the |@| has category code 0 (i.e. it starts macros) when reading the file.
%    
%    After execution of |\article| or any of the other similar commands,
%    |\CP@ParseArg| contains the argument of the \BibTeX{} entry. 
%    Later on, when parsing the argument, we must throw away the \BibTeX{} keyword
%    and the comma which it seperates from the interesting \meta{key}|=|\meta{value}
%    list. This is the reason why |\CP@ParseArg| is initialized with |dummy,|.
%    If the \BibTeX{} file does not exist, we report an error.
%    \begin{macrocode}
\gdef\CP@ParseArg{dummy,}
\ifCP@nobib
\else
  \IfFileExists{\jobname.BibTeX.txt}{%
    \bgroup
      \catcode`\@0
      \bgroup
        \def\inproceedings##1{\xdef\CP@ParseArg{##1}}%
        \def\article##1{\xdef\CP@ParseArg{##1}}%
        \def\incollection##1{\xdef\CP@ParseArg{##1}}%
        \def\misc##1{\xdef\CP@ParseArg{##1}}%
        \input{\jobname.BibTeX.txt}%
      \egroup
    \egroup
  }{%
    \PackageError{CoverPage}{No BibTeX file (\jobname.BibTeX.txt) found!}%
      {you must provide a \jobname.BibTeX.txt file}%
  }
\fi
%    \end{macrocode}
%    
%    
%    \subsection{Interface setup (with \texttt{keyval})}
%
%    We then define a series of \texttt{keyval} keys which will be used by both 
%    \BibTeX{} parser and user interface. The first commands are for automatic 
%    creation of source data.
%    
%    The source field is defined as "in: \meta{source}", followed by a language-specific
%    reference to the \BibTeX{} entry if the \BibTeX{} class defines defines a 
%    main source of reference. Predefined are: |booktitle| (e.g. for |@inproceedings|) and
%    |journal| (for |@article|).
%
%    \begin{macrocode}
\def\CP@SourceGiven{\gdef\CP@InSource{\CP@In\ \CP@Source. \CP@BibRef}}
\define@key{CoverPage}{booktitle}{\gdef\CP@Source{#1}\CP@SourceGiven}
\define@key{CoverPage}{journal}{\gdef\CP@Source{#1}\CP@SourceGiven}
%    \end{macrocode}
%    
%    The next five keys define the five default text strings.
%    
%    \begin{macrocode}
\define@key{CoverPage}{title}{\gdef\CP@Title{#1}}
\define@key{CoverPage}{author}{\gdef\CP@Author{#1}}
\define@key{CoverPage}{institute}{\gdef\CP@Institute{#1}}
\define@key{CoverPage}{insource}{\gdef\CP@InSource{#1}}
\define@key{CoverPage}{copyright}{\gdef\CP@Copyright{#1}}
%    \end{macrocode}
%    
%    Then we check whether we have enough information to create the copyright 
%    notice automatically. Also note that we introduce a macro |CPPublisherCheck|;
%    experienced \LaTeX{} users may use this to set up further publisher--copyright
%    tests like the ones which are predefined.
%    
%    \begin{macrocode}
\define@key{CoverPage}{year}{\gdef\CP@Year{#1}}
\define@key{CoverPage}{publisher}{%
  \edef\CP@Publisher{#1}%
  \def\CP@Compare{Springer}%
  \ifx\CP@Publisher\CP@Compare
    \gdef\CP@Copyright{\textcopyright\ \CP@Year\ Springer-Verlag.\\
      See also LNCS-Homepage: \url{http://www.springeronline.com/lncs}}%
  \else
    \def\CP@Compare{IEEE}%
    \ifx\CP@Publisher\CP@Compare
      \gdef\CP@Copyright{\textcopyright\ \CP@Year\ IEEE. Personal use of this 
        material is permitted. However, permission to reprint/republish this 
        material for advertising or promotional purposes or for creating new 
        collective works for resale or redistribution to servers or lists, or 
        to reuse any copyrighted component of this work in other works must 
        be obtained from the IEEE.}%
    \else
      \def\CP@Compare{SPIE}%
      \ifx\CP@Publisher\CP@Compare
        \gdef\CP@Copyright{\textcopyright\ \CP@Year\ Society of Photo-Optical 
          Instrumentation Engineers. This paper was published in \CP@Source\ 
          and is made available as an electronic reprint with permission of 
          SPIE. One print or electronic copy may be made for personal use only. 
          Systematic or multiple reproduction, distribution to multiple 
          locations via electronic or other means, duplication of any material 
          in this paper for a fee or for commercial purposes, or modification 
          of the content of the paper are prohibited.}%
      \else
        \def\CP@Compare{Kluwer}%
        \ifx\CP@Publisher\CP@Compare
          \gdef\CP@Copyright{\textcopyright\ \CP@Year\ Kluwer Academic Publishers}%
        \else
          \CPPublisherCheck
        \fi
      \fi
    \fi
  \fi
}
\def\CPPublisherCheck{\relax}
%    \end{macrocode}
%
%
%    \subsection{Parsing and interpretation of the bibliography file}
%
%    We parse the \BibTeX{} file using a two step macro expanding mechanism to eliminate 
%    the \BibTeX{} keyword.
%    One should also mention that we (locally) disabled errors for unknown keys in 
%    \texttt{keyval}. The macro |\KV@errx| which is used here is no part of the official
%    \texttt{keyval} user interface, but it is the only way to do it; we do not want to
%    disable \texttt{keyval} error checking in general.
%    \begin{macrocode}
\gdef\CPProcessBibEntry{%
    \edef\temp{\noexpand\CP@killBibTeXKeyword\CP@ParseArg\par}%
    \def\CP@killBibTeXKeyword##1,##2\par{##2}%
    \edef\tempii{\noexpand\setkeys{CoverPage}{\temp}}%
    \bgroup
        \def\KV@errx##1{\relax}% keyval: ignore unknown keys
        \tempii
    \egroup
}
%    \end{macrocode}
%
%
%    \subsection{Printing of cover page}
%
%    The printing of a page whose page layout is different from the remaining document 
%    is a difficult job. We solve this by going deep down to \TeX{} internals, namely 
%    the |\shipout| procedure which is called every time a page is assembled 
%    completely (which is always done in the internal box register 255). We just compile 
%    our own box 255 and give it to |\shipout|. Note the shift of the upper left 
%    corner; this is due to the fact that the reference point is 1 inch from left and top
%    paper border.
%    \begin{macrocode}
\newcommand{\CP@PrintCoverPage}{\bgroup\global\setbox255\vbox{%
    \vspace*{-.25in}%
    \noindent\hspace*{-.25in}%
    \rlap{\begin{minipage}[t][\CP@CoverPageHeight][c]{\CP@CoverPageWidth}%
        {\CoverPageHeader}%
        \vfill%
        {\CoverPageBody}%
        \vfill%
        {\CoverPageFooter}%
    \end{minipage}}%
}\egroup\shipout\box255}
%    \end{macrocode}
%
% This function is automatically called at |\begin{document}|. Additionally, we
% advance a page in two-sided mode.
%
%    \begin{macrocode}
\AtBeginDocument{\CP@PrintCoverPage\CP@AdvancePageTwoSided}
%    \end{macrocode}
%
%
%
%
%
%    \section{Implementation Part III -- User Interface}
%
%    All macros explained in the `Usage' section can be found here.
%    The user can change fonts, has access to the \texttt{keyval} interface and can 
%    redefine cover page composition. Change with |\renewcommand| if you want to.
%    
%
%    \subsection{Set fonts}
%
%    We first define fonts for the five main text areas.
%
%    \begin{macrocode}
\newcommand{\CPTitleFont}{\Huge\sffamily\bfseries\upshape}
\newcommand{\CPAuthorFont}{\LARGE\sffamily\mdseries\upshape}
\newcommand{\CPInstituteFont}{\Large\sffamily\mdseries\upshape}
\newcommand{\CPInSourceFont}{\large\sffamily\mdseries\upshape}
\newcommand{\CPCopyrightFont}{\large\sffamily\mdseries\upshape}
%    \end{macrocode}
%    
%    
%    \subsection{Access to \texttt{keyval} interface}
%    
%    Pass arguments on to \verb!\setkeys! from \texttt{keyval} package.
%    \begin{macrocode}
\newcommand{\CoverPageSetup}[1]{\setkeys{CoverPage}{#1}}
%    \end{macrocode}
%    
%    
%    \subsection{Definition of cover page}
%    
%    The header is empty by default:
%    \begin{macrocode}
\newcommand{\CoverPageHeader}{}
%    \end{macrocode}
%    
%    The body consists of: [rule] - title --- author - institute --- source - [rule]
%    
%    \begin{macrocode}
\newcommand{\CoverPageBody}{%
    \parbox{\linewidth}{%
        \centering%
        \rule{\linewidth}{1.5pt}\par
        \vspace*{2mm}
        {\CPTitleFont\CP@Title\par}
        \vspace{5mm}
        {\CPAuthorFont\CP@Author\par}
        \vspace{1mm}
        {\CPInstituteFont\CP@Institute\par}
        \vspace{5mm}
        {\CPInSourceFont\CP@InSource\par}
        \vspace*{2mm}
        \rule{\linewidth}{1.5pt}\par
    }%
}
%    \end{macrocode}
%    
%    The footer contains the \BibTeX{} entry, the copyright notice and info/logo
%    at the bottom.  Footer info is some document info; the logo is empty by default.
%    
%    \begin{macrocode}
\newcommand{\CoverPageFooter}{%
    \normalfont
    \ifCP@nobib
    \else
        \noindent\textsf{\large \BibTeX:}
        {\small\verbatiminput{\jobname.BibTeX.txt}\par}%
        \vspace{5mm}%
    \fi
    {\CPCopyrightFont\CP@Copyright\par}
    \vspace{10mm}
    {\CoverPageFooterInfo\hfill\CoverPageFooterLogo}
}
\newcommand{\CoverPageFooterInfo}{%
    \parbox[b]{.5\textwidth}{%
        \tiny document created on: \today\\
        created from file: \texttt{\jobname.tex}\\
        cover page automatically created with \texttt{CoverPage.sty}\\
        (available at your favourite CTAN mirror)%
    }%
}
\newcommand{\CoverPageFooterLogo}{}
%    \end{macrocode}
%
%
%
%
%
%    \subsection{Start actions}
%    
%    At the end of this package, we process \BibTeX{} entry and read the config file. 
%    The order is important: if both contain e.g. an entry for \texttt{institute}, the 
%    last setting overrides the first one. The config file is given the higher priority 
%    here.
%
%    \begin{macrocode}
\CPProcessBibEntry
\IfFileExists{CoverPage.cfg}%
    {%%
%% CoverPage.ins
%% Batch file, run through LaTeX. 
%% Installs CoverPage.sty from CoverPage.dtx
%%
%% Copyright (C) 2006 by Matthias M�hlich
%%                       (matthias.muehlich@lfb.rwth-aachen.de)
%% 
%% This file may be distributed and/or modified under the
%% conditions of the LaTeX Project Public License, either version 1.2
%% of this license or (at your option) any later version.
%% The latest version of this license is in:
%%
%%    http://www.latex-project.org/lppl.txt
%%
%% and version 1.2 or later is part of all distributions of LaTeX 
%% version 1999/12/01 or later.
%%
\input docstrip.tex
\keepsilent
\usedir{tex/latex/LfBTools}
\preamble

LaTeX package for generating a cover page for scientific papers. This 
cover page contains bibliographical information and a copyright notice.

Copyright (C) 2006 Matthias M�hlich

This file may be distributed and/or modified under the conditions of
the LaTeX Project Public License, either version 1.2 of this license
or (at your option) any later version.  The latest version of this
license is in:

   http://www.latex-project.org/lppl.txt

and version 1.2 or later is part of all distributions of LaTeX version
1999/12/01 or later.

\endpreamble
%%\askforoverwritefalse
\generate{%
    \file{CoverPage.sty}{\from{CoverPage.dtx}{CoverPage}}
    \nopreamble\nopostamble
    \file{CoverPage.cfg}{\from{CoverPage.dtx}{CoverPageCFG}}
}
\Msg{**************************************************************}
\Msg{*}
\Msg{* To finish the installation you have to move the following}
\Msg{* files into a directory searched by LaTeX:}
\Msg{*}
\Msg{*  \space\space CoverPage.sty}
\Msg{*  \space\space CoverPage.cfg}
\Msg{*}
\Msg{* You should adapt the configuration file "CoverPage.cfg"}
\Msg{* to your needs.}
\Msg{*}
\Msg{* To produce the documentation run the following file}
\Msg{* through LaTeX:}
\Msg{*}
\Msg{*  \space\space CoverPage.dtx}
\Msg{*}
\Msg{* Happy TeXing!}
\Msg{*}
\Msg{**************************************************************}
\endbatchfile
}%
    {\PackageWarning{CoverPage}{No config file (CoverPage.cfg) found!}}
%</CoverPage>
%    \end{macrocode}
%
%    The |%</CoverPage>| guard signals the end of the \texttt{CoverPage.sty} code.
%
%
%
%
%
%    \section{Config File}
%
%    The code for the sample config file is also stored in \texttt{CoverPage.dtx}.
%    It is written to \texttt{CoverPage.cfg} including all comments and without
%    preamble or postamble from \texttt{docstrip} such that it is easy to understand
%    for the human user.
%
%    \begin{macrocode}
%<*CoverPageCFG>
%%%%%%%%%%%%%%%%%%%%%%%%%%%%%%%%%%%%%%%%%%%%%%%%%%%%%%%%%%%%%%%%%%%%%%%%%%%%%
%%
%%  Config file for ExtraTitlePage
%%
%%  This file "CoverPage.cfg" is read by "CoverPage.sty" 
%%  (if it exists). Use it to define default settings, e.g. for your 
%%  institute or your group logo.
%%  
%%%%%%%%%%%%%%%%%%%%%%%%%%%%%%%%%%%%%%%%%%%%%%%%%%%%%%%%%%%%%%%%%%%%%%%%%%%%%

%% set institute
%% (the only information which cannot be extracted from BibTeX entry)
\CoverPageSetup{institute = {Institute of Applied Nonsense}}

%% redefine header, body, or footer with \renewcommand
\renewcommand{\CoverPageHeader}{%
  \LARGE\bfseries%
  \dotfill{} This is my header \dotfill}
\renewcommand{\CoverPageFooterLogo}{%
  \Huge\sffamily%
  L-O-G-O}


%% These are "our" settings:
%%\CoverPageSetup{institute = {Institute of Imaging and Computer Vision\\
%%   RWTH Aachen University, 52056 Aachen, Germany\\
%%   tel: +49 241 80 27860, fax: +49 241 80 22200\\
%%   web: www.lfb.rwth-aachen.de}}
%%\renewcommand{\CoverPageHeader}{\includegraphics[width=\textwidth]{lfbheader}}
%%\renewcommand{\CoverPageFooterLogo}{\includegraphics[width=.35\textwidth]{rwthlogo}}
%</CoverPageCFG>
%    \end{macrocode}
%    \Finale
\endinput
